\documentclass{article}

% if you need to pass options to natbib, use, e.g.:
% \PassOptionsToPackage{numbers, compress}{natbib}
% before loading nips_2017
%
% to avoid loading the natbib package, add option nonatbib:
% \usepackage[nonatbib]{nips_2017}

\usepackage{nips_2017}

% to compile a camera-ready version, add the [final] option, e.g.:
% \usepackage[final]{nips_2017}

\usepackage[utf8]{inputenc} % allow utf-8 input
\usepackage[T1]{fontenc}    % use 8-bit T1 fonts
\usepackage{hyperref}       % hyperlinks
\usepackage{url}            % simple URL typesetting
\usepackage{booktabs}       % professional-quality tables
\usepackage{amsfonts}       % blackboard math symbols
\usepackage{ntheorem}       % blackboard math symbols
\usepackage{nicefrac}       % compact symbols for 1/2, etc.
\usepackage{microtype}      % microtypography

\usepackage{url}
\usepackage{todonotes}
\usepackage{amsbsy}
\usepackage{amsmath}
%\usepackage{amsthm}
\usepackage{amsmath}%
\usepackage{MnSymbol}%
\usepackage{wasysym}%
\usepackage[cal=dutchcal,
calscaled=1,
bb=boondox,
scr=euler]{mathalfa}

\usepackage[capitalise]{cleveref}
\crefformat{equation}{(#2#1#3)}

\newcommand{\ie}[0]{\emph{i.e.},~}
\newcommand{\eg}[0]{\emph{e.g.},~}
\newcommand{\aka}[0]{a.k.a.~}
\newcommand{\etal}{\emph{et al.~}}
\newcommand{\etc}{\emph{etc.~}}
\newcommand{\wrt}{w.r.t.~}
\def\todo#1{{\color{red}{{todo:\ #1\ }}}}
\newcommand{\defeq}{\ensuremath{\doteq}}

\newcommand{\mat}[1]{\ensuremath{\mathbf{#1}}}
\newcommand{\gr}[1]{\ensuremath{\mathcal{#1}}}
\newcommand{\set}[1]{\ensuremath{\mathbb{#1}}}
\renewcommand{\vec}[1]{\ensuremath{\overrightarrow{#1}}}
\newcommand{\prm}[1]{\ensuremath{^{(#1)}}}
\newcommand{\ttt}[1]{\ensuremath{^{(#1)}}}
\newcommand{\grn}[2]{\ensuremath{\gr{#1}\prm{#2}}}
\newcommand{\Trp}[0]{\ensuremath{^{\mathsf{T}}}}
\newcommand{\XX}[0]{\ensuremath{\mat{X}}}
\newcommand{\YY}[0]{\ensuremath{\mat{Y}}}
\newcommand{\ZZ}[0]{\ensuremath{\mat{Z}}}
\newcommand{\xx}[0]{\ensuremath{\mat{x}}}
\newcommand{\yy}[0]{\ensuremath{\mat{y}}}
\newcommand{\zz}[0]{\ensuremath{\mat{z}}}
\newcommand{\Xset}[0]{\ensuremath{\set{X}}}
\newcommand{\Yset}[0]{\ensuremath{\set{Y}}}
\newcommand{\Rset}[0]{\ensuremath{\set{R}}}
\renewcommand{\Re}[0]{\ensuremath{\set{R}}}

\newtheorem{theorem}{Theorem}[section]
\newtheorem{lemma}[theorem]{Lemma}
\newtheorem{proposition}[theorem]{Proposition}
\newtheorem{observation}[theorem]{Observation}
\newtheorem{claim}[theorem]{Claim}
\newtheorem{objective}[theorem]{Objective}
\newtheorem{corollary}[theorem]{Corollary}
\theoremstyle{definition}
\newtheorem{definition}{Definition}[section]
\newtheorem{example}{Example}[section]
\theoremsymbol{\ensuremath{\blacksquare}}
\newtheorem*{proof}{Proof}%[section]

\title{Exchangeable Tensor Convolution}

% The \author macro works with any number of authors. There are two
% commands used to separate the names and addresses of multiple
% authors: \And and \AND.
%
% Using \And between authors leaves it to LaTeX to determine where to
% break the lines. Using \AND forces a line break at that point. So,
% if LaTeX puts 3 of 4 authors names on the first line, and the last
% on the second line, try using \AND instead of \And before the third
% author name.

\author{
  David S.~Hippocampus\thanks{Use footnote for providing further
    information about author (webpage, alternative
    address)---\emph{not} for acknowledging funding agencies.} \\
  Department of Computer Science\\
  Cranberry-Lemon University\\
  Pittsburgh, PA 15213 \\
  \texttt{hippo@cs.cranberry-lemon.edu} \\
  %% examples of more authors
  %% \And
  %% Coauthor \\
  %% Affiliation \\
  %% Address \\
  %% \texttt{email} \\
  %% \AND
  %% Coauthor \\
  %% Affiliation \\
  %% Address \\
  %% \texttt{email} \\
  %% \And
  %% Coauthor \\
  %% Affiliation \\
  %% Address \\
  %% \texttt{email} \\
  %% \And
  %% Coauthor \\
  %% Affiliation \\
  %% Address \\
  %% \texttt{email} \\
}

\begin{document}
% \nipsfinalcopy is no longer used

\maketitle

\begin{abstract}
....
\end{abstract}
\section{Introduction}
We derive this for matrices and then generalize to tensors.
Also for now consider a single input/output channel.
Lets add relevant papers to the bibliography as we proceed.

\section{Notation}
\begin{itemize}
\item $\XX \in \Re^{N \times M}$
\item $\xx \in \Re^{N M}$: vectorized $\XX$, also denoted by $\vec{\XX}$.
\item $\grn{S}{N}$ symmetric group of \textit{all} permutations of $N$ objects
\item $\grn{G}{N} = \{\grn{g}{N}_i\}_i$ a permutation group of $N$ objects 
\item $\grn{g}{N}_i$ or $\mat{G}_i\prm{N}$ both can refer to the matrix form of the permutation $\grn{g}{N}_i \in \grn{G}{N}$.
\end{itemize}

\section{Problem}
$\grn{S}{N} \times \grn{S}{M} < \grn{S}{N M}$ is the product of group of all permutations of $N$ and $M$ objects. It has $N! M!$ members compared to $(NM)!$ members of the group $\gr{S}_{N M}$.
The members $\grn{g}{N,M} \in \grn{S}{N}$ can be represented by matrices $\mat{G}\prm{N M}$. This matrix ``acts'' on the vector $\vec{\XX}$.
Alternatively we can represent them using the pair $(\mat{G}\prm{N}, \mat{G}\prm{M})$ where the ``action'' of this pair is defined on the matrix $\XX$:
\begin{align*}
  \mat{G}\prm{N M \times N M} \vec{\XX} \equiv \vec{\mat{G}\prm{N} \XX \mat{G}\prm{M}}
\end{align*}

\begin{objective}\label{obj:1}
  Define a parameter matrix $\Theta \in \Re^{ N M \times N M}$ such that
  \begin{itemize}
  \item \textbf{Equivariance} (adding nonlinearity does not change anything):
  \begin{align}
    \mat{G}\prm{N M} \Theta \vec{\XX} = \Theta \mat{G}\prm{N M} \vec{\XX}\quad \forall \vec{\XX} \in \Re^{N M}, \mat{G}\prm{N M} \in \grn{S}{N} \times \grn{S}{M}
  \end{align}
\item \textbf{No equivariance wrt any other permutation}:
  For any permutation of elements of the matrix $\XX$ that is not produced using a permutation of rows and columns $\mat{H} \in \grn{S}{N M}, \mat{H} \notin \grn{S}{N} \times \grn{S}{M}$, the
  parameter matrix $\Theta$ should have the property that
  \begin{align}
   \exists \XX \quad s.t. \quad  \mat{H}\prm{N M} \Theta \vec{\XX} \neq \Theta \mat{H}\prm{N M} \vec{\XX} 
  \end{align}
  These two necessary and sufficient conditions of equivariance strictly define the form of $\Theta$.
\end{itemize}
\end{objective}


\section{A Proposal}
Here is a proposal for $\Theta \in \Re^{NM \times NM}$:
\begin{align}
  \Theta_{(m,n), (m',n')} \defeq
  \begin{cases}
    \theta_1 & m = m', n=n'\\
    \theta_2 & m = m', n \neq n' \\
    \theta_3 & m \neq m', n = n' \\
    \theta_4 & m \neq m', n \neq n' \\
  \end{cases}
\end{align}
Intuitively, each element ${m,n}$ at the output interacts with 1) corresponding input element; 2) all elements of the input in its row; 3) its column; 4) all other elements of $\XX$.

We can write the product $\Theta \vec{X}$ as
\begin{align}\label{eq:param}
  \Theta * \XX \defeq \Theta \vec{X} = \theta'_1 \XX +  \theta'_2 \mat{1} (\mat{1}\Trp \XX) + \theta'_3 (\XX \mat{1})\mat{1}\Trp + \theta'_4 (\mat{1}\Trp \XX \mat{1})
\end{align}
where $\mat{1} = [1,\ldots,1]\Trp$ is the a column vector of appropriate size.

Replacing the summation with average we get:
\begin{align}\label{eq:param_avg}
  \Theta * \XX \defeq \Theta \vec{X} = \theta''_1 \XX + \frac{\theta''_2}{N} \mat{1} (\mat{1}\Trp \XX) + \frac{\theta''_3}{M} (\XX \mat{1})\mat{1}\Trp + \frac{\theta''_4}{N M} (\mat{1}\Trp \XX \mat{1})
\end{align}

It should not be hard to show that this matrix commutes with any permutation of rows and columns (Equivariance condition in the objective). It follows as a Corollary of Proposition 3.1 in \citep{ravanbakhsh_symmetry}. However, it is not obvious whether the other direction (non-equivariance) holds or not. If it does not, it may still work well in practice.

\begin{theorem}\todo{prove}
  Parameter matrix $\Theta$ of \ref{eq:param} satisfies the conditions of Objective~\ref{obj:1} (??)
\end{theorem}

In practice, we substitute the summation with either average or max operation. This does not change equivariance properties.
\todo{multiple channels}

\section{Extension to Tensors}

\section{Applications}
\subsection{Tensor Completion}\todo{find applications}

\subsection{Matrix Completion}
Given partially observed tensor $\XX \in \Re^{M \times N \times K}$, where $K$ is the number of input-channels, we seek to estimate the unobserved values.
Exchangeability is only within rows and columns (not channels).
We use $\set{O} = \{(m,n)\}$ denote the set of observed indices. Here we have the option of 1) producing the output for \textit{all} entries ($m,n$)
in \cref{eq:param} starting from the first layer, or 2) produce it only at the final layer or 3) only produce the output for entries $\set{Q} = \{m,n\}$
where a query is made.
Note that depending on this choice the layer \cref{eq:param} would change to only ``use'' the observed entries $\set{O}$ and ``produce'' the required ones $\set{Q}$.

In collaborative filtering applications, the rows ($n$) represent customers and columns ($m$)
represent products. In addition to customer/product ratings we may have feature-vectors $\yy \in \Re^{N \times K'}$ and
$\zz \in \Re^{M \times K''}$ for each customer and product. Assuming $K' = K'' = 1$, the output of \cref{eq:param} is extended to incorporate
these using $\theta_5 \yy \mat{1}\Trp + \theta_5 \mat{1} \zz\Trp$. For multiple input/output channels this needs to be extended.
We also have the option of defining the layer such that we also have outputs corresponding to $\yy$ and $\zz$, in addition to the matrix output of $\XX$.

Consider a deep model where the output of the final layer is $\YY \in \Re^{N \times M \times K}$,
where $K=1$ shows that the last layer has a single channel.
The loss function minimized through back-propagation is
$\sum_{(m,n) \in \set{O}} \ell(X_{(m,n)}, \YY_{(m,n)})$, where $\ell$ is an appropriate matching loss, such as $L_2$ loss.

\section{ToDo List}
Theoretical side:
\begin{enumerate}
\item Try to prove the theorem. Does it satisfy the second part of Objective? 
\item If not see whether a better parameter-sharing scheme exists.
\item Write it for multiple channels
\item Extend to tensors using a readable notation
\end{enumerate}

Implementation:
\begin{enumerate}
\item Implement this layer for dense input/output matrices
\item Implement it for Sparse input output, when the sets $\set{O}$ and $\set{Q}$ are given
\item We should be able to switch in/out max and mean
\item Apply it to benchmark datasets (\eg movie-lens?) for matrix completion
\end{enumerate}


\bibliographystyle{abbrvnat} % sorted alphabetically, labeled with numbers
{\bibliography{refs.bib}} % names file keylatex.bib as my bibliography file
\end{document}
